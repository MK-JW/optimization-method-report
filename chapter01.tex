\section{概述}

\subsection{一维搜索与无约束优化方法概述}

一维搜索方法是一类在一个固定方向或区间内寻找目标函数最优解的优化方法。其基本思想是在目标函数的一个维度上进行搜索,逐步逼近最优解。通常,目标函数是多维的,但在一维搜索中,我们只关心其中的一个维度或方向。这个方法的核心任务是通过计算和评估目标函数值来寻找该方向上的最优点。目前常见的精确一维搜索方法包括对分搜索法、等间隔搜索法、斐波那契法与黄金分割法,常见的非精确一维搜索方法包括Armijo搜索方法,wolfe搜索方法以及强wolfe搜索方法。

无约束优化方法是指在没有任何约束条件下,寻找一个目标函数最小值或最大值的方法。在无约束优化问题中,优化的目标是找到函数的最优解,而不需要考虑任何限制或约束条件。通常,这类优化问题的目标函数是一个实值函数,定义在一个连续的变量空间内。无约束优化问题常见于机器学习、数据拟合、图像处理等领域。目前常见的无约束优化方法包括梯度下降法、powell法、共轭梯度法与拟牛顿法。

一维搜索方法是无约束优化算法中的重要工具,尤其在确定合适步长时发挥关键作用。无约束优化方法通过计算梯度、海塞矩阵等信息来确定搜索方向,而一维搜索可以确定在某一方向上的精确步长或可接受步长,从而确保每步都向最优解靠近。因此,一维搜索在无约束优化中通常是不可或缺的组成部分,尤其在处理大规模、高维度的优化问题时,能够有效提高求解效率和精度。
\vspace{1cm}
\begin{table}[h]
\centering
\begin{tabular}{|lll|}
\hline
\multicolumn{3}{|c|}{常见一维搜索方法}  \\ \hline
\multicolumn{1}{|l|}{精确一维搜索方法}  & \multicolumn{2}{l|}{\begin{tabular}[c]{@{}l@{}}对分搜索法、等间隔搜索法\\ 、黄金分割法、斐波那契法\\ 、牛顿法\end{tabular}}     \\ \hline
\multicolumn{1}{|l|}{非精确一维搜索方法} & \multicolumn{2}{l|}{\begin{tabular}[c]{@{}l@{}}Armijo法、goldenstein法\\ 、wolfe法、强wolfe法\end{tabular}} \\ \hline
\multicolumn{3}{|c|}{常见无约束优化方法} \\ \hline
\multicolumn{3}{|l|}{\begin{tabular}[c]{@{}l@{}}坐标轮换法、powell法、梯度下降法、共轭梯度法\\ 、拟牛顿法\end{tabular}}                                       \\ \hline
\end{tabular}
\caption{常见一维搜索与无约束优化方法}
\end{table}