\section{我的学习感悟与体会}

从一维搜索到无约束约束优化方法,这部分是优化方法中较为基础的部分。本报告列举了常见的一维搜索与无约束优化方法。

其中,在编写无约束优化方法代码的过程中,我认为对于一维搜索方法确定其上下界是一个非常重要的过程,但是对于精确一维搜索方法目前感觉没有一个较好的方法来确定其上下界,对于无已知条件的函数来说,利用划线法或者尝试修改区间可以确定初始的上下区间,但是可能对于有多个极小点的函数来说,可能会陷入局部的最优步长,这个问题目前比较困难。

对于非精确步长,其本身就是确定一个可接受步长,所以其也可能在有多个极值的条件下可能确定的也为局部的最优步长,并且对于不同步长的选择可能影响其收敛速率,如果区间给的不合理会导致收敛困难。

对于无约束优化方法,方向的确定是十分重要的,而目前方向的确定大都需要利用到梯度信息,在没有梯度信息的情况下坐标轮换法、powell法以及改进powell法对于初始点的坐标是严格的,这就会导致最终求解较为困难。
